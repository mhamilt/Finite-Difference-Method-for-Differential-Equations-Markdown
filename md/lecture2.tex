\newcommand{\dif}{\mathop{}\!\mathrm{d}}
\newcommand{\al}{\alpha}
\newcommand{\bt}{\beta}
\newcommand{\om}{\omega}
\newcommand{\gm}{\gamma}
\newcommand{\lb}{\lambda}
\newcommand{\lm}{\lambda_-}
\newcommand{\lp}{\lambda_+}
\newcommand{\thw}{\theta_w}
\newcommand{\thp}{\theta_\phi}
\newcommand{\tht}{\theta}

\newcommand{\etp}{e_{t+}}
\newcommand{\etm}{e_{t-}}

\newcommand{\esp}{e_{x+}}
\newcommand{\esm}{e_{x-}}

\newcommand{\dsp}{\delta_{x+}}
\newcommand{\dsm}{\delta_{x-}}
\newcommand{\dsd}{\delta_{x\cdot}}
\newcommand{\dss}{\delta_{xx}}
\newcommand{\dssss}{\delta_{xxxx}}

\newcommand{\dtp}{\delta_{t+}}
\newcommand{\dtm}{\delta_{t-}}
\newcommand{\dtd}{\delta_{t\cdot}}
\newcommand{\dtt}{\delta_{tt}}
\newcommand{\dtttt}{\delta_{tttt}}
\newcommand{\dspm}{\delta_{x\pm}}

\newcommand{\mtp}{\mu_{t+}}
\newcommand{\mtm}{\mu_{t-}}
\newcommand{\mtd}{\mu_{t\cdot}}
\newcommand{\mtt}{\mu_{tt}}

\newcommand{\norm}[1]{\left\|#1\right\|}
\newcommand{\innp}[1]{\left\langle#1\right\rangle}
\newcommand{\virg}[1]{``#1''}


\chapter{The Simple Harmonic Oscillator}\label{chap:SHO}

Simple harmonic motion is obtained under linear condition, that is
\begin{equation}
    \phi = \frac{Kx^2}{2}.
\end{equation}
Under such choice, \eqref{eq:SHO} becomes
\begin{equation}\label{eq:SHOreal}
    \frac{d^2 x}{dt^2} = - \omega_0^2 x,
\end{equation}
where the radian natural frequency $\omega_0 = \sqrt{K/m}$ was introduced. There are multiple reasons to be wanting to study the simple harmonic oscillator here: a variety of mechanical systems can be approximated by \eqref{eq:SHOreal}, see e.g. Fig. \ref{fig:oscExamples}. Furthermore, as will be seen in later lectures, distributed systems may themselves be approximated as a bank of parallel oscillators, each one corresponding to one ``mode'' of vibration. Finally, the simple harmonic oscillator possesses exact (i.e. closed-form) solutions, both analytically and numerically, and is thus a very useful test case for the numerical schemes that will be introduced below. It was seen that the solution is expressed as the sum of two periodic functions, i.e.
\begin{equation}
    x(t) = a \cos(\omega_0 t) + b \sin{\omega_0 t}.
\end{equation}
The constants $a,b$ are uniquely determined from the intial conditions. Setting $x(t=0)=x_0$ and $\frac{dx}{dt}(x=0)=v_0$, one gets
\begin{equation}\label{eq:SHOexact}
    x(t) = x_0 \cos(\omega_0 t) + \frac{v_0}{\omega_0}\sin(\omega_0 t)
\end{equation}
The energy components are obtained explictly as
\begin{equation}
    E_k = \frac{m}{2}\left(x_0\omega_0\sin(\omega_0 t) - v_0 \cos(\omega_0 t) \right)^2, \quad E_p = \frac{K}{2}\left( x_0 \cos(\omega_0 t) + \frac{v_0}{\omega_0}\sin(\omega_0 t) \right)^2.
\end{equation}
Summing the two expressions together, one gets
\begin{equation}
    H(t) = \frac{m v_0^2}{2} + \frac{K x_0^2}{2} = H_0,
\end{equation}
i.e. \eqref{eq:EnCons}. 
\begin{figure}
    \includegraphics[width=\linewidth]{Figures/OscillatorsExamples.pdf}
    \caption{Examples of harmonic oscillators. The mass-spring system (a), and the series RLC circuit (b) are usually given as examples of harmonic oscillators. The case of a cantilever beam with a point mass (c), and the pendulum (d) may  be approximated as harmonic oscillators in the case of small vibrations.}\label{fig:oscExamples}
\end{figure}



% \subsubsection{Stability}

% The stability of the model problem \eqref{eq:SHOreal} was already shown via energy arguments, in Sec. \ref{sec:BoundsSHO}. It can be appreciated that energy analysis handles equally well linear and nonlinear cases.  For the simple harmonic oscillator, stability may be drawn using frequency domain techniques as well, as anticipated in Sec. \ref{sec:FreqDomAn}. Thus, assuming a trial solution of the form $x = e^{st}$ gives
% \begin{equation}
% s^2  x = - \omega_0^2  x \quad \rightarrow \quad s = \pm j \omega_0
% \end{equation}
% In this case, $s$ is purely imaginary and the solutions are oscillating.  


\section{A finite difference scheme}

Consider the time series $x^n$, approximating the true solution $x(t)$ of \eqref{eq:SHOreal}. As a first example of a working finite difference scheme, consider
\begin{equation}\label{eq:Scheme1}
    \dtt x^n = -\omega_0^2 x^n.
\end{equation}
Expanding out the operator, one gets 
\begin{equation}\label{eq:FDbasic}
    x^{n+1} = x^n(2-\omega_0^2 k^2) - x^{n-1},
\end{equation}
and hence the update requires one multiply and one sum. Clearly, it is convenient to store the value $2-\omega_0^2 k^2$ offline, so that one need not  recompute this value at each time step. The update \eqref{eq:FDbasic} is \emph{explicit}: by this term, we denote a scheme such that
\begin{equation}
    x^{n+1} = f(x^n,x^{n-1}), \qquad \text{(explicit scheme)}
\end{equation}
We will soon encounter schemes that are instead \emph{implicit}, i.e. 
\begin{equation}\label{eq:ImplSchemeDef}
    g(x^{n+1})= f(x^n,x^{n-1}), \qquad \text{(implicit scheme)}
\end{equation}
where $g$ is generally a nonlinear function in  $x^{n+1}$.


Though scheme \eqref{eq:Scheme1} looks reasonable, there is no guarantee (for the moment) that the computed solutions are indeed an approximate form of the true solution $x(t)$. In some cases, as will be seen shortly, the time series computed by $\eqref{eq:Scheme1}$ diverges, in some other cases, it remains bounded. The next few sections will explain the idea of \emph{convergence}, and the closely linked idea of \emph{stability}. A notion of stability may be formalised as follows. We say that a scheme is stable in stability region $\Lambda \subseteq \mathbb{R}^+$ if, for any positive time constant $\tau \leq n k$, there exist a positive index $M$ such that
\begin{equation}\label{eq:StabDef}
    |x^n| \leq C_\tau \sum_{m = 0}^M |x^m|
\end{equation}
for a constant $C_\tau > 0$ independent of $k \in \Lambda$. In practice, stability is defined as a bound on the time series including the first $M+1$ steps. 

\subsection{Stability via frequency domain analysis}\label{sec:FreqDomSHO}

As anticipated earlier, frequency-domain techinques may be employed in the analysis of stability of linear, time-invariant discrete systems, such as \eqref{eq:Scheme1}. For that, an \emph{ansatz} of the form \eqref{eq:ansatz} is substituted:
\begin{equation}
    x^n = \hat x z^{n}.
\end{equation}
In this equation, notation is a little mixed-up, since on the left-hand side $n$ denotes the time index, whereas on the right-hand side it is an exponent! In practice, we keep the same apex notation in both cases, for the sake of notation, but the meaning is very different. We get
\begin{equation}
    \hat x z^n\left( z - (2-\omega_0 k^2) + z^{-1}\right) = 0, \quad \rightarrow \quad z_{\pm} = \frac{2 - \omega_0^2 k^2 \pm \omega_0^2 k^2\sqrt{1 - \frac{4}{\omega_0^2 k^2}}}{2}.
\end{equation}
Hence, the solution is
\begin{equation}\label{eq:sol_z_SHO}
    x^n =  A_+ z^n_+ + A_- z^n_-,
\end{equation}
for complex constants $A_\pm$. We assume that the scheme is started using two starting values $x^0, x^1$ (obtained from $x_0$, $v_0$ of the continuous problem). Then
\begin{equation}
    x^0 = A_+ + A_-, \quad x^1 = A_+ z_+ + A_- z_-.
\end{equation}
From these, the complex constants are obtained as
\begin{equation}\label{eq:ApAm}
    A_+ = \frac{x^0z_- - x^1}{z_- - z_+},\,\,\, A_- = \frac{x^1 - x^0z_+ }{z_- - z_+}.
\end{equation}
If the square root in $z_\pm$ is a real number, than $z_-$ has magnitude larger than unity, and the solution $x^n$ will therefore grow exponentially over time: this is \emph{instability}. On the other hand, when the square root in imaginary, then $z_\pm$ become oscillating, and $z_\pm$ are complex conjugates. This condition is obtained as
\begin{equation}\label{eq:StabCondSHO}
    k < \frac{2}{\omega_0}, 
\end{equation}
which is an upper bound on the time step, once the natural frequency of the oscillator is set. In the case of oscillating solutions, one has
\begin{equation}
    z_\pm = r e^{\pm j \theta},
\end{equation}
with $r = 1$, and $\tan\theta = \left(\omega_0^2k^2\sqrt{\frac{4}{\omega_0^2 k^2}-1}\right)/(2-\omega_0^2 k^2)$.

To check stability, definition \eqref{eq:StabDef} is applied, to give
\begin{equation}\label{eq:SHObound} 
    |x^n| = |A_+ r^n e^{j \theta n} + A_- r^n e^{-j \theta n}| \leq |A_+| + |A_-| \leq \frac{|x^0|+|x^1|}{|\sin \theta|} \triangleq C_\tau \sum_{m=0}^1 |x^m|,
\end{equation}
and thus the absolute value of the solution at the time $n>1$ is bounded in terms of the values at $n=0,1$, with bounding constant $C_\tau = 1/|\sin\theta|$. (The first inequality in the above was obtained via the triangle inequality. Then, the fact that $|r|=|e^{\pm j\theta}|=1$ was used, and finally the values from \eqref{eq:ApAm} were substituted in). A numerical check of the current bound is given in Fig. \ref{fig:SHObounds}.
\begin{figure}
    \includegraphics[width = \linewidth]{Figures/boundsSHO.png}
    \caption{Simple Harmonic Oscillator. Numerical check on bound \eqref{eq:SHObound}. The sample rate used in the examples is $f_s = 2000$ Hz. Starting values are $x^0 = x^1 = 1$. Dashed line line is bound \eqref{eq:SHObound}, characteristic frequency as indicated, with $\omega_{max}=2/k$.}\label{fig:SHObounds}
\end{figure}
Note that the same condition may be arrived at via \emph{von Neumann} analysis. Recall the DTFT of the $\dtt$ operator, as per \eqref{eq:DTFTdtt}. Transforming \eqref{eq:Scheme1} in the frequency domain accordingly, one gets
\begin{equation}
\left(-\frac{4}{k^2}\sin^2\left( \frac{\omega k}{2}\right) + \omega_0^2\right)\mathcal{X}\left\{ x^n\right\} = 0.
\end{equation}
One must impose  $0 \leq \sin^2\left( \frac{\omega k}{2}\right) \leq 1$, which is possible if and only if \eqref{eq:StabCondSHO} holds. If this condition is violated, one has that the frequency $\omega$ becomes pure imaginary, thus the sine becomes a hyerbolic sine, with unbounded growth (instability).




\subsubsection{Stability via energy analysis}



The discussion of Sec. \ref{sec:EnAnGen} suggests that, if the model problem can be shown to have a conserved total energy, with non-negative kinetic and potential terms, then the solution can be bounded in terms of the energy. It may be tempting to try to find a discrete version of \eqref{eq:EnAnGen}, valid in the discrete case. To that end, \eqref{eq:Scheme1} is multiplied by $m \dtd x^n$, to get
\begin{equation}\label{eq:SHOen1}
    m \dtd x^n (\dtt x^n +\omega_0^2 x^n) = 0
\end{equation}
A couple of useful identities (that will be used throughout) are given here:
\begin{equation}\label{eq:IdsFD}
    \dtd x^n \, \dtt x^n = \dtp \left( \frac{(\dtm x^n)^2}{2}\right), \,\, \dtd x^n \, x^n = \dtp \left( \frac{x^n \etm x^n}{2}\right).
\end{equation}
These are proven by simple algebra. Using these identities in \eqref{eq:SHOen1}, one gets
\begin{equation}
    \dtp \left( \frac{m}{2}{(\dtm x^n)^2} + \frac{K}{2} {x^n \etm x^n} \right) = 0,
\end{equation} 
that is a discrete counterpart of \eqref{eq:En2}. Multiplication by $m$ was here used so to yield units of energy in the expression within the brakets, though of course one may obtain conserved energy per unit mass via multiplication by $\dtd x^n$ alone. It is easy, in the above, to recognise a discrete approximation to the continuous energy balance \eqref{eq:En1}. In this case, one may define an \emph{interleaved} time series $\mathfrak{h}^{n-1/2}$, corresponding to the discrete conserved energy:
\begin{equation}\label{eq:DiscEnSHO}
    \mathfrak{h}^{n-1/2} \triangleq \frac{m}{2}{(\dtm x^n)^2} + \frac{K}{2} {x^n \etm x^n} = \mathfrak{h}^{1/2}.
\end{equation}
In light of the discussion in the continuous case, one may of course use energy conservation as a means to bound the growth of solutions over time. The problem here, is that $\mathfrak{h}^{n-1/2}$ may \emph{not} be positive, since the potential energy is of indefinite sign. Instances leading to negative energy overall are a manifestation of instability, and must be avoided. It may be useful, then, to bound the potential term in the energy expression. Using
\begin{equation}
    x^n \etm x^n  =  (\mtm x^n)^2 - \frac{k^2}{4}\left(\dtm x^n \right)^2,
\end{equation}
the total energy is 
\begin{equation}\label{eq:ModEnSHO}
    \mathfrak{h}^{n-1/2} = \frac{m\left(\dtm x^n \right)^2}{2} \left(1 - \frac{\omega_0^2 k^2}{4}\right) + \frac{K (\mtp x^n)^2}{2} \geq \frac{m\left(\dtm x^n \right)^2}{2} \left(1 - \frac{\omega_0^2 k^2}{4}\right),
\end{equation}
and thus the total energy will be non-negative if and only if \eqref{eq:StabCondSHO} is satisfied. Fig. \ref{fig:EnConsSHO} shows the energy components and error for scheme \eqref{eq:Scheme1}. The energy components are given as per \eqref{eq:DiscEnSHO}, and it is remarked that the potential term \emph{does} become negative at times. It is the \emph{overall} energy that is positive. Of course, the equivalent expression in \eqref{eq:ModEnSHO} has modified expressions for the kinetic and potential energies, that are always non-negative under stability condition \eqref{eq:StabCondSHO}.
\begin{figure}
    \includegraphics[width=\linewidth,clip,trim={5cm 0.0cm 5cm 0cm}]{Figures/EnergyErr.png}
    \caption{Energy behaviour of scheme \eqref{eq:Scheme1}. Left: energy components. Kinetic (dashed), potential (dash-dotted), total (solid). Right: energy error $\mathfrak{h}^{n-1/2}/\mathfrak{h}^{1/2}-1$. The oscillator is initialised with $x_0=v_0=1$, and $\omega_0=100$ rad/s.}\label{fig:EnConsSHO}
\end{figure}




\subsubsection{Consistency and accuracy}

We are now introducing the idea of   \emph{local truncation error} (LTE), denoted here $\varepsilon^n$. Applying the finite difference scheme to the true solution $x(t)$ yields a definiton of the LTE as
\begin{equation}\label{eq:LTEdef}
    \dtt x(t_n) + \omega_0^2 x(t_n) = \varepsilon^n.
\end{equation}
Using Taylor series arguments, as per \eqref{eq:Errs}, one gets
\begin{equation}
    \left( \frac{d^2 x(t)}{dt} + \omega_0^2 x(t)\right)|_{t=t_n} + O(k^2) = \varepsilon^n,
\end{equation}
and since $x(t)$ is the true solution, one recovers $\varepsilon^n = O(k^2)$. The behaviour of the LTE as a function of $k$ describes the idea of \emph{consistency}: a scheme is said to be consistent if
\begin{equation}
    \lim_{k\rightarrow 0}\varepsilon^n = 0.
\end{equation}
In practice, consistent schemes are such that the local error becomes small as $k$ is decreased. Usually, $\varepsilon = O(k^p)$, and one may conclude that the scheme is $p^{th}$-order accurate. Of course, this is not entirely true, since the question of accuracy is tightly bound to the ideas of stability and convergence: higher-accurate schemes may \emph{never} converge for a given model problem. The idea of accuracy is only going to be meaningful when a scheme is provably stable in some manner. As an example, consider a fourth-order accurate difference operator discretising the second time derivative:
\begin{equation}\label{eq:FourthOrderdtt}
    \bar\delta_{tt} x(t) = \left(\frac{-\etp^2 + 16 \etp - 30 + 16\etp - \etp^2}{12k^2}\right) x(t) = \frac{d^2 x}{dt^2} + O(k^4).
\end{equation}
Though technically ``higher'' accurate, this approximation is always unstable, even for the simple problem of a free particle ($\phi = 0$ in \eqref{eq:PhiF}). Using the test solution $x^n = \hat x z^n$ for this test case, one has
\begin{equation}
    \hat x z^n \left(-z^2 + 16 z - 30 + 16 z^{-1} - z^{-2}\right) = 0,
\end{equation}
and it is easy to verify that there exist one (real) root $z \approx 13.9282$ for which clearly $|z|>1$. The scheme is unstable, and the higher accuracy of error of $\bar\delta_{tt}$ has no real advantage. 


\subsubsection{Convergence}

In turn, what we are really interested in is the evolution of the global error $E^n$, which must remain bounded. For stable, consistent schemes, the global error $E^n$ can be expected to maintain the same trend as the local truncation error $\varepsilon^n$, though this claim would require a formal proof. As an example, the output of scheme \eqref{eq:Scheme1} is compared against the exact solution given in \eqref{eq:SHOexact}. The initial conditions in the continuous system are set as $x_0 =1$, $v_0 = 0$, giving $x(t) = \cos(\omega_0 t)$. The initial conditions in the discrete scheme are given as $x^0 = x_0 = 1$, $x^1 = \cos(\omega_0 k)$ (which discretises the exact solution at the time $t=k$). Fig. \ref{fig:SHOerrs} (a double log plot) shows that indeed slopes of 2 are recovered.


\begin{figure}
    \includegraphics[width = \linewidth]{Figures/OscError.pdf}
    \caption{Global error of scheme \eqref{eq:Scheme1}. Here, $nk = 1$. Initial conditions are given as $x_0 = 1$, $v_0 = 0$, giving $x(t) = \cos(\omega_0 t)$. The numerical initial conditions are $x^0 = 1$, $x^1 = \cos(\omega_0 k)$. The three lines correspond to $\omega_0 = 100$ rad/s (solid), $\omega_0 = 200$ rad/s (dashed), $\omega_0 = 300$ rad/s (dash-dotted). Sample rate $f_s = 2000$ Hz.}\label{fig:SHOerrs}
\end{figure}



The behaviour of the global error as a function of the time step $k$ encapsulates the idea of convergence. A scheme is convergent if 
\begin{equation}\label{eq:convDef}
    \lim_{k\rightarrow 0} (x(t_n)-x^n) = \lim_{k\rightarrow 0} E^n = 0.
\end{equation}
In practice, as the time step is decreased, the global error goes to zero. We will come back to the idea of convergence later on. In general, a stable and consistent method is also convergent (this should be proven rigourously, and we will postpone this discussion until later sections).  


\subsubsection{Initialisation}\label{sec:Init}
In the previous subsection, the scheme was initialised exactly using knowledge coming from the exact solution $x(t)$ as per \eqref{eq:SHOexact}. Of course, generally an exact solution is not available, and schemes must be initialised in some other manner. Since we usually know the initial position and velocity of the oscillator (in continuous time) $x_0, v_0$, we would like to know how to use this information to extract suitable initial values for the time series, i.e. $x^0$, $x^1$.
\begin{figure}[hbt]
    \centering
    \includegraphics[width = 1\linewidth]{Figures/OscErrorOrder.pdf}
    \caption{Global error of scheme \eqref{eq:Scheme1}. Initialisation with first-order accurate (markers) and second-order accurate (lines) initial conditions. Here, $nk = 1$. Initial conditions are given as $x_0 = 0.5$, $v_0 = 0.5$. The numerical initial conditions are $x^0 = x_0$, $x^1 = kv_0 + x_0 - 0.5 k^2 \omega_0^2 x_0$ (lines), $x^0 = x_0$, $x^1 = kv_0 + x_0$ (markers). The three cases correspond to $\omega_0 = 100$ rad/s, $\omega_0 = 200$ rad/s, $\omega_0 = 300$ rad/s.}\label{fig:SHOerrsOrders}
\end{figure}
Obviously, one can set 
\begin{equation}
    x^0=x_0
\end{equation}
For $x^1$, one possible solution is to use a simple forward difference to compute it from $v_0$ and $x^0$, i.e. 
\begin{equation}
    \dtp x^0 = v_0 \,\,\, \rightarrow \,\,\, x^1 = x^0 + kv_0.
\end{equation}
This approximation to the initial conditions is only \emph{first-order accurate.} This is easily proven via Taylor series arguments. One may be tempted then to use the centered difference $\dtd$, instead of the forward difference $\dtp$, since it yields a second-order accurate approximation to the time derivative. Of course, this is not possible directly, since the stencil of $\dtd$ is too large: we do not know what $x^{-1}$ is (in fact, this value is not defined). However, consider the following expression for the centered time difference:
\begin{equation}\label{eq:dtdInit}
    \dtd = \dtp - \frac{k}{2} \dtt = \frac{d}{dt} + O(k^2).
\end{equation}
Of course, the value of $\dtt$ is not substituted directly, rather via \eqref{eq:Scheme1}, i.e. $\dtt = -\omega_0^2$. Thus, a second-order accurate intial condition can be given as
\begin{equation}
    \dtd x^0 = v_0 \,\,\, \rightarrow \,\,\, x^1 = x^0 + kv_0 - \frac{k^2}{2}\om_0^2x^0.
\end{equation}
Fig. \ref{fig:SHOerrsOrders} shows the error plots, computed against the exact solution \eqref{eq:SHOexact}, displaying the expected  trends in the limit of high sample rate. Note that, for lower values of the sample rate, the error of the first-order accurate scheme may in fact be lower than the second-order accurate scheme, though in the limit of vanishing time step the correct trends are recovered, and convergence is of course faster for the second-order accurate schemes.  

Higher order accurate approximations are of course possible. Here is a list, obtained using Taylor-series arguments. 
\begin{align}\label{eq:HigherOrderICsOscillator}
\dtp x^0 &= v_0 \quad \text{first order}\\
\left(\dtp -\frac{k}{2}\dtt \right) x^0 &= v_0 \quad \text{second order}\\
\left(\dtp -\frac{k}{2}\dtt - \frac{k^2}{6}\dtp \dtt \right) x^0 &= v_0 \quad \text{third order}\\
\left(\dtp -\frac{k}{2}\dtt - \frac{k^2}{6}\dtp \dtt - \frac{k^3}{24}\dtt^2 \right) x^0 &= v_0 \quad \text{fourth order}
\end{align}
In the expressions above, substituting $(\dtt)^p = (-\omega_0)^p$ gives a way to compute $x^1$, knowing $x^0$ and $v_0$. 


\subsection{Frequency warping and modified equation techniques}\label{eq:ModEqTechniques}

The discussion about accuracy has so far dealt with the idea of order-accuracy, i.e. how the global error $E^n$ behaves as the time step $k$ is decreased. It was seen that finite difference scheme usually behave in such a way that $|E^n| = O(k^p)$, where $p \geq  1$ is the order of accuracy. This is, of course, one way of looking at how well a scheme performs. For the oscillator, it may be useful to measure the degree of accuracy in the frequency domain. Solution \eqref{eq:sol_z_SHO} suggests that the solutions are oscillating with natural frequency
\begin{equation}\label{eq:ErrFreqs}
    \omega = \frac{1}{k}\arctan{\frac{\omega_0^2k^2\sqrt{\frac{4}{\omega_0^2 k^2}-1}}{(2-\omega_0^2 k^2)}} = \omega_0 \left(1 + \frac{\omega_0^2 k^2}{24} +  \frac{3 \omega_0^4 k^4}{640} + O(\omega_0^6k^6) \right),
\end{equation} 
and hence the natural frequency computed by  scheme \eqref{eq:Scheme1} is second-order accurate compared to the natural frequency of the continuous system. See also left panel of Fig. \ref{eq:FreqWarping}. The error in frequency can be quite audible, as $\omega_0$ approaches the limit of stability $\omega_{max}=2/k$. In practice, the cent deviation from \eqref{eq:ErrFreqs} is given by
\begin{equation}
    1200 \log_{2}\frac{\omega}{\omega_0} = 1200\log_2\left(1 + \frac{\omega_0^2 k^2}{24} +  \frac{3 \omega_0^4 k^4}{640} + O(\omega_0^6k^6) \right), 
\end{equation}
and this is shown in the right panel of Fig. \ref{eq:FreqWarping}.
\begin{figure}
    \includegraphics[width=\linewidth,clip,trim={1cm 0.0cm 1cm 0cm}]{Figures/FreqWarp.pdf}
    \caption{Frequency warping (left) and cent deviation (right) of scheme \eqref{eq:Scheme1}. In the left panel, three natural frequencies are shown: 100 rad/s (solid), 200 rad/s (dahsed), 300 rad/s (dash-dotted). The right panel shows cent deviation up to $O(\omega_0^6k^6)$ (solid), and exact (dashed).}\label{eq:FreqWarping}
\end{figure}
The frequency warping effects are quite evident. It may be preferable, then, to construct schemes with a higher accuracy. This, of course, cannot be done by merely using difference operators with a larger stencil, such as the one given in \eqref{eq:FourthOrderdtt}: this would result in unstable behaviour! A different approach, known as \emph{modified equation method}, can be constructed starting from Taylor-series arguments. One has
\begin{equation}\label{eq:dttExpasion}
    \dtt = \sum_{l=1}^{\infty}\frac{2k^{2(l-1)}}{(2l)!}\frac{d^{2l}}{dt^{2l}}= \frac{d^2}{dt^2} + \frac{k^2}{12}\frac{d^4}{dt^4} + \frac{k^4}{360}\frac{d^6}{dt^6} + O(k^6).
\end{equation}
Considering for the moment the expansion up to the term $l=2$, it is natural to add a term $-\frac{k^2}{12}\dtt \dtt x^n$ on the left-hand side of \eqref{eq:Scheme1}, in order to cancel the $O(k^2)$ error. Then, one may use $\dtt = -\omega_0^2$ twice, to get 
\begin{equation}
    \dtt x^n = \frac{1}{k^2}\left(-\omega_0^2k^2 + \frac{\omega_0^4 k^4}{12} \right) x^n,
\end{equation}
and of course the local truncation error $\varepsilon^n$ is now $O(k^4)$. Considering now the next term in the series, proportional to $k^4$, and using $\dtt = -\omega_0^2$ three times, one gets
\begin{equation}
    \dtt x^n = \frac{1}{k^2}\left(-\omega_0^2k^2 + \frac{\omega_0^4 k^4}{12} - \frac{\omega_0^6k^6}{320}\right) x^n,
\end{equation}
and this approximation is $O(k^6)$. Of course, one may go on and add more and more terms to the series. The expansion can be rewritten conveniently as
\begin{equation}
    \frac{1}{k^2}\left(-\omega_0^2k^2 + \frac{\omega_0^4 k^4}{12} - \frac{\omega_0^6 k^6}{320} + ...\right) = \frac{2}{k^2}\left(-1 + \underbrace{1 - \frac{\omega_0^2k^2}{2} + \frac{\omega_0^4k^4}{4!} - \frac{\omega_0^6k^6}{6!} + ...}_{\cos(\omega_0k)} \right),
\end{equation}
and thus, adding infinite terms to the series results in
\begin{equation}\label{eq:SHOexactNum}
    \dtt x^n =  \frac{2}{k^2}\left(-1 + \cos(\omega_0 k) \right)x^n.
\end{equation}
Since the series expansion converges to a known function in this case, we can say that scheme \eqref{eq:SHOexactNum} solves \eqref{eq:SHOreal} \emph{exactly}. Of course, this is somewhat too strong a statement: following the discussion in Sec. \ref{sec:Init}, we know that the scheme is only going to be as accurate as its initial conditions in this case. However, scheme \eqref{eq:SHOexactNum} does not introduce errors in the frequency domain. 


\section{Loss}\label{sec:LossSHO}

\begin{figure}[hbt]
    \centering
    \includegraphics[width=0.9\linewidth,clip]{Figures/PhaseSpLoss.png}
    \caption{Time evolution and phase portraits of lightly damped oscillator ((a),(b)), and overdamped oscillator ((c),(d)). The natural frequency is $\omega_0=100$ rad/s, and the decay times are $\tau_{60}=5$ s (lightly damped), 0.05 s (overdamped). Initial conditions are $x_0=-0.01$ m, $v_0=0.04$ m/s.}\label{fig:dampedOsc}
\end{figure}
We are now going to study the behaviour of the oscillator in the presence of dissipative forces. These are always present in some form in real systems. Terms proportional to the velocity are usually a good starting point to model viscous damping. For the oscillator, a modification of \eqref{eq:SHOreal} can then be given as
\begin{equation}\label{eq:SHOLoss}
    \frac{d^2 x}{dt^2} = -\omega_0^2 x - 2c \frac{dx}{dt}.
\end{equation}
Here, $c \geq 0$ is a loss parameter (assumed constant, and measured in s$^{-1}$). We remark that \eqref{eq:SHOLoss} is still a linear, time invariant system, and we may infer its stability properties via \emph{ansatz} \eqref{eq:ansatz}. Thus, 
\begin{equation}\label{eq:LossAnsatz}
    \hat x e^{st}\left(s^2 + 2c s + \omega_0^2 \right) = 0, \,\, \text{implying}\,\, s_{\pm} = -{c} \pm \sqrt{{c}^2 - \omega_0^2}.
\end{equation}
The qualitative behaviour of the oscillator will depend on the value of $c$ compared to $\omega_0$, i.e. whether the square root in the expression for $s_\pm$ is real or pure imaginary. Two cases of interest may be extracted as follows:
\begin{enumerate}
    \item $c<\omega_0$. In this case, the oscillator is only lightly damped, and $s_\pm = -c \pm j \sqrt{\omega_0^2-c^2}$. Remembering the definition of $s$ in \eqref{eq:LapT}, one may extract $\sigma = - c$, $\omega = \sqrt{\omega_0^2-c^2}$. The solution to \eqref{eq:SHOLoss} may then be written as $x(t) = A_+ e^{s_+ t} + A_- e^{s_- t} = e^{-c t}\left(A_+ e^{j \omega t} + A_- e^{-j\omega t} \right)$. As per usual, the complex constants $A_+,A_-$ are determined from the intial conditions. The solution is in this case the product of an oscillating solution, times an exponentially damped envelope. The frequency of vibration is $\omega = \omega_0\sqrt{1 - c^2/\omega_0^2}\approx \omega_0\left(1 - \frac{c^2}{2\omega_0^2} \right)$ and is thus lower than the natural frequency of the undamped oscillator. In this case, motion is not strictly periodic, since the mass is never really going to extend as far at each oscillation because of the energy given away to losses. However, it still makes sense to speak of period of vibration, as $\tau = 2\pi / \omega.$ 
    \item $c>\omega_0$. In this case, $s_\pm$ are both real, and negative, i.e. $s_\pm < 0$. Thus, the solutions are still ``physical'' in that they die out exponentially as time increases. However, the mass does not oscillate. This case is sometimes referred to as \emph{overdamped oscillator.}
\end{enumerate}
Fig. \ref{fig:dampedOsc} shows illustrative examples of such cases. Note that, in phase space, the orbits spiral toward the centre, and motion is strictly speaking not periodic.
A third case (\emph{critically damped}) is obtained whenever $c=\omega_0$. This case serves as a mathematical ``boundary'' between the overdamped and lightly damped cases, and is never really realised in practice. A useful quantity to quantify loss is the \emph{decay time} $\tau_{60}$. This is defined as the time taken by the oscillator to reduce its amplitude of vibration by 60 dB. In the lightly damped case, the amplitude envelope is simply $e^{-ct}$. Thus, the implicit definition of $\tau_{60}$ is obtained as
\begin{equation}\label{eq:tau60}
    -60 = 20 \log_{10}e^{-c\tau_{60}}, \,\, \text{and upon inversion: } \,\, \tau_{60} = \frac{3}{c} \ln(10).
\end{equation}
This shows that the loss constant $c$ is most easily interpreted as a function of the decay time.



\subsection{Energy analysis}


Qualitative results on stability can be obtained via energy analysis. Multiplying \eqref{eq:SHOLoss} by $m \frac{dx}{dt}$ and using the same indentities as for \eqref{eq:En1}, we get
\begin{equation}\label{eq:EnBalLoss}
    \frac{d}{dt}\left( \frac{m}{2} \left(\frac{dx}{dt}\right)^2 + \frac{K x^2}{2}   \right) = -Q(t) \triangleq - 2mc \left( \frac{dx}{dt} \right)^2 \leq 0,
\end{equation}
implying that the total energy is not increasing. Here, $Q(t) \geq 0$ is the power dissipated by the oscillator. Thus, bounds \eqref{eq:bnds}  hold here as well. It is somewhat harder to draw more quantitative results here, without knowledge on the form of $x(t).$ However, it is easy to draw trajectories in phase space: instead of a closed loop, the mass now spirals toward the centre as a result of losses. 


\subsection{A finite difference scheme}

As a discretisation for \eqref{eq:SHOLoss} is obtained as
\begin{equation}\label{eq:FDSHOLoss}
    \dtt x^n = -\omega_0^2 x^n -2c \dtd x^n,
\end{equation}
yielding an update equation
\begin{equation}
    \left(ck+1\right) x^{n+1} = (2-\omega_0^2 k^2) x^n + (ck - 1)x^n.
\end{equation}
Using the definition of local truncation error $\varepsilon^n$, as per \eqref{eq:LTEdef}, one gets $\varepsilon^n = O(k^2)$, $\forall n$, showing that the LTE is second-order accurate. Stability may be inferred using either frequency-domain analysis, or energy methods. Application of the former via the ansatz $x^n = \hat x z^n$ gives
\begin{equation}\label{eq:SolFdLoss}
    \hat x z^n \left((1+ck)z - (2-\omega_0^2 k^2) -(ck-1)z^{-1}\right), \,\, z_\pm = \frac{2-\omega_0^2k^2 \pm \sqrt{(2-\omega_0^2k^2)^2 - 4(1-ck)(1+ck)}}{2(1+ck)},
\end{equation}
and, using the Schur-Cohn condition \eqref{eq:SchurCohnStab}, $|z_\pm|<1$ if and only if
\begin{equation}
    k < \frac{2}{\omega_0},
\end{equation}
that is the same as \eqref{eq:StabCondSHO}. The same condition may be arrived at via energy analysis. To that end, multiply \eqref{eq:FDSHOLoss} by $\dtd x^n$, 
\begin{equation}
    \dtd x^n \, \dtt x^n = - \dtd x^n \, \omega_0^2 x^n - 2c (\dtd x^n)^2.
\end{equation}
Using identities \eqref{eq:IdsFD}, and multiplying by the mass $m$ to restore units of energy, one gets
\begin{equation}
    \dtp \left( \frac{m}{2}{(\dtm x^n)^2} + \frac{K}{2} {x^n \etm x^n} \right) = - 2mc (\dtd x^n)^2 \leq 0,
\end{equation}
which is a discrete counterpart of \eqref{eq:EnBalLoss}. Thus, the discrete energy is non-increasing, and when the total energy is itself non-negative, boundedness of the solution results. Thus, the stability condition \eqref{eq:StabCondSHO} is recovered in this case as well. Of course, \eqref{eq:SHObound}  holds in this case too. The numerical decay time may be established via knowledge of the solutions $z_\pm$ in \eqref{eq:SolFdLoss}. Assuming oscillating behaviour, $z_\pm$ are complex conjugates with absolute value
\begin{equation}
    |z_\pm| = \sqrt{\frac{1-ck}{1+ck}}.
\end{equation}
Thus, the numerical decay time index $n_{60}$ is given by
\begin{equation}
    -60 = 20\log_{10}\left({\frac{1-ck}{1+ck}}\right)^{n_{60}/2},\,\,\,\, n_{60}k = \frac{6k \ln(10)}{\ln \frac{1+ck}{1-ck}}\approx \tau_{60} - c k^2 \ln(10),
\end{equation}
showing that the numerical decay time (in seconds) is $O(k^2)$ compared to the exact decay time $\tau_{60}$ defined in \eqref{eq:tau60}.

\subsection{Higher-order schemes}
Higher-order accurate schemes may of course be obtained in this case as well. To that end, consider the definition of the LTE, as
\begin{equation}
    \left(\dtt+2c\dtd \right) x(t_n) = - \omega_0^2 x(t_n) + \varepsilon^n,
\end{equation}
where $x(t_n)$ is the true solution. Expanding in a Taylor series, one has
\begin{equation}
    \left(\frac{d^2}{dt^2} + 2c \frac{d}{dt} \right)\left(1 + \frac{k^2}{6}\frac{d^2}{dt^2}\right)x(t_n) - \frac{k^2}{12}\frac{d^4}{dt^4}x(t_n) + O(k^4) = - \omega_0^2 x(t_n) + \varepsilon^n.
\end{equation}
This suggests the use the following modified scheme, in order to cancel the terms proportional to $k^2$:
\begin{equation}
    \left(\dtt + 2c \dtd \right)\left(1 - \frac{k^2}{6}\dtt\right)x^n + \frac{k^2}{12}\dtt\dtt x^n = -\omega_0^2 x^n.
\end{equation}
Since $\dtt + 2c \dtd  = -\omega_0^2 + O(k^2)$, the scheme above can be written as (to the order $O(k^4)$):
\begin{equation}
    \left(\dtt + 2c \dtd \right)x^n - \frac{k^2}{6}(-\omega_0^2) \dtt x^n + \frac{k^2}{12}\dtt \dtt x^n = -\omega_0^2 x^n.
\end{equation}
The hard bit left is to find a suitable approximation to $\dtt \dtt$, involving at most a stencil of width 2. This can be accomplised in the following way
\begin{equation}\label{eq:dttsq}
    \dtt \dtt \approx \left(-\omega_0^2 \etm -2c \dtm \right)\left(-\omega_0^2 \etp -2c \dtp \right) = \omega_0^4 + 4c \omega_0^2 \dtd + 4c^2 \dtt.
\end{equation}
Putting it all together, one obtains a fourth-order accurate approximation to \eqref{eq:SHOLoss} as
\begin{equation}
    \left(1 + \frac{k^2}{6}(\omega_0^2 + 2c^2) \right)\dtt x^n = -\omega_0^2\left(1 + \frac{\omega_0^2 k^2}{12} \right) x^n - 2c \left(1 + \frac{\omega_0^2k^2}{6} \right)\dtd x^n.
\end{equation}
Higher-order accurate schemes can may be obtained this way, i.e. finding approximations to $\dtt^p$, involving only operators of width 2. A sketch of the idea is given briefly here. From \eqref{eq:dttsq}, one may construct $\dtt^3$ in the following way:
\begin{align*}
    \dtt\dtt\dtt \approx \left( \omega_0^4 + 4c \omega_0^2 \dtd + 4c^2 \dtt\right)\dtt \approx                                              
    \left( \omega_0^4\etm + 4c \omega_0^2 \dtm + 4c^2 (-\omega_0^2\etm -2c\dtm)\right)\dtt \approx                                          \\\left( \omega_0^4\etm + 4c \omega_0^2 \dtm + 4c^2 (-\omega_0^2\etm -2c\dtm)\right)(-\omega_0^2\etp -2c\dtp)= \\
    (-\omega_0^6 + 4c^2 \omega_0^4) + \left(-6 c \omega_0^4 + 16 c^3 \omega_0^2 \right)\dtd + \left( -8 c^2 \omega_0^2 + 16 c^4\right)\dtt, 
\end{align*}
showing that $\dtt^3$ can be approximated using a stencil of width 2. One may of course use the modified equation technique described above to any desired order. Luckily, the oscillator with loss also posseses an exact solution, where ``exact'' is intended in the same way as for \eqref{eq:SHOexact} (i.e. exact up to the accuracy order of the intial conditions). Considering again the continuous equation with loss, \eqref{eq:SHOLoss}, under the following transformation
\begin{equation}
    X(t) = e^{ct} x(t),
\end{equation}
one gets
\begin{equation}
    \frac{d^2X}{dt^2} + \left(\omega_0^2-c^2 \right) X = 0.
\end{equation}
Thus, the exact scheme \eqref{eq:SHOexactNum} for the undamped oscillator can be applied to the transformed variable $X$. When transformed back to $x$, this gives
\begin{equation}\label{eq:SHO2}
    % \left(\dtt + \frac{2e^{-c k}}{k^2} \left(e^{c k}-\cos\left(\sqrt{\omega_0^2-c^2} k\right)\right) + \frac{e_{t-}}{k^2} (e^{-2c k}-1)\right)x^n = 0.
    \dtt x^n = \left(-\frac{2}{k^2}\left(1 - \cos\left((\sqrt{\omega_0^2-c^2} \, k\right) \right)- \frac{\etp(e^{ck}-1) + \etm(e^{-ck}-1)}{k^2}\right)x^n
\end{equation}
This scheme solves \eqref{eq:SHOLoss} exactly, in particular, the frequency of oscillation and the numerical decay time are exact.



\section{Forced Oscillations}

The equation of the oscillator including loss and source terms is given as
\begin{equation}\label{eq:SHOForced}
    \frac{d^2 x}{dt^2} = -\omega_0^2 x - 2c \frac{dx}{dt} + f(t),
\end{equation}
where $f(t)$ is a time-dependent force per unit mass. Energy analysis leads here to the following energy balance
\begin{equation}
\frac{d}{dt}\left( \frac{m}{2} \left(\frac{dx}{dt}\right)^2 + \frac{K x^2}{2}   \right) = - Q(t) + P(t),
\end{equation}
where $Q(t) = 2mc \left(\frac{dx}{dt}\right)^2$ is the dissipated power, and where $P(t) = m\frac{dx}{dt}f(t)$ is the injected power.




In this case, the system is still linear, but it is not time invariant. Solutions via transform techniques can be obtained, involving the use of the one-sided Laplace transform \eqref{eq:LapT} so to incorporate the effects of the initial conditions and  of the external forcing. In this case, substitution of the simpler \emph{ansatz} \eqref{eq:ansatz} is not possible, because of the presence of the forcing term. Considering $t\geq 0$, the application of the one-sided Laplace transform in \eqref{eq:SHOForced} gives
\begin{equation}
    \int_0^{\infty} \left( \frac{d^2 x}{dt^2} + \omega_0^2 x + 2c \frac{dx}{dt} - f(t)\right)e^{-st} \, \dif t = 0.
\end{equation}
Using integration by parts, one gets (remember that we defined $x_0 = x(0), v_0 = dx(0)/dt$):
\begin{equation}
    \hat x(s)\left( s^2 + 2cs + \omega_0^2\right) = (s+2c)x_0 + v_0 + \hat{f}(s),
\end{equation}
and thus
\begin{equation}
    \hat x(s) = \frac{(s+2c)x_0 + v_0 + \hat{f}(s)}{s^2 + 2cs + \omega_0^2}.
\end{equation}
The expression above may be decomposed into the \emph{transient} and \emph{forced response}. The transient is given by the contribution of the initial conditions only, without external forcing, so:
\begin{equation}
    \hat x(s) = \hat x_{tr}(s) + \hat x_{fr}(s) = \frac{(s+c)x_0 + (v_0 + cx_0)}{s^2 + 2cs + \omega_0^2} + \frac{\hat{f}(s)}{s^2 + 2cs + \omega_0^2}.
\end{equation}
This shows that the contributions of the transient and of the forced response are independent of each other: they add up in the final response. The solution in the time domain is obtained upon inversion of $\hat x(s).$  It is best to write the denominator of $\hat x(s)$ as $(s+c)^2 + \left(\sqrt{\omega_0^2-c^2}\right)^2$, since this is the form reported in \eqref{eq:LaplTtable}.
% \begin{align}
% \mathcal{L}^{-1}\left\{\frac{1}{(s+c)^2 + (\sqrt{\omega_0^2-c^2})^2}\right\} &= \frac{e^{-ct}}{\sqrt{\omega_0^2-c^2}}\sin\left(\sqrt{\omega_0^2-c^2}\,\,t\right), \\
% \mathcal{L}^{-1}\left\{\frac{s}{(s+c)^2 + (\sqrt{\omega_0^2-c^2})^2}\right\} &= e^{-ct}\left(\cos\left(\sqrt{\omega_0^2-c^2}\,\,t\right) - \frac{c}{\sqrt{\omega_0^2-c^2}}\sin\left(\sqrt{\omega_0^2-c^2}\,\,t\right) \right). 
% \end{align}
Using these (with $a = \sqrt{\omega_0^2-c^2}$), the solution to the transient is
\begin{equation}
    x_{tr}(t) = e^{-ct}\left(x_0 \cos\left(\sqrt{\omega_0^2-c^2}\,\,t\right) + \frac{v_0 + c x_0}{\sqrt{\omega_0^2-c^2}}\sin\left(\sqrt{\omega_0^2-c^2}\,\,t\right)\right),
\end{equation}
which is of course the same as \eqref{eq:LossAnsatz} (we did not go through the substitution of the initial conditions there, but the result is the same). For the forced response, we may employ the \emph{convolution} property of the Laplace transform, which states that
\begin{equation}
    \mathcal{L}^{-1}(\hat a(s)\hat b(s)) = \int_{0}^t a(t-u)b(u) \, \dif u,
\end{equation}
and thus, using $\hat a = 1/((s+c)^2 + (\sqrt{\omega_0^2-c^2})^2)$, $\hat b = \hat f$, one gets
\begin{equation}\label{eq:steadyStSHO}
    x_{fr}(t) = \int_0^t \frac{e^{-c(t-u)}}{\sqrt{\omega_0^2-c^2}}\sin\left(\sqrt{\omega_0^2-c^2}\,\,(t-u)\right) f(u) \, \dif u. 
\end{equation}
This integral is not generally computable analytically. However, it also encapsulates the idea that the forced response to \emph{any} forcing can be obtained as the convolution with the \emph{impulse response}. To that end, considering $f(t)=\delta(t-t_0)$, with delta being Dirac's delta here, and denoting as per usual the initial time by $t_0$, one gets from \eqref{eq:steadyStSHO}:
\begin{equation}\label{eq:GreenSHO}
    G(t|t_0) = \frac{e^{-c(t-t_0)}}{\sqrt{\omega_0^2-c^2}}\sin\left(\sqrt{\omega_0^2-c^2}\,\,(t-t_0)\right), \,\,\,\, t\geq t_0,
\end{equation}
and $G$ is zero for $t<t_0$. The symbol $G(t|t_0)$ denotes the \emph{Green's function} of the harmonic oscillator, i.e. the  response to a Dirac impulse at $t=t_0$. In particular, for zero initial conditions, the forced response is also the total response of the system. 
\begin{figure}
    \centering
    \includegraphics[width=0.9\linewidth]{Figures/ForcedOsc.png}
    \caption{Time evolution of forced oscillator. Dashed-dotted line is response to initial conditions, dahsed line is Green's function \eqref{eq:GreenSHO}, solid line is total response. The oscillator is activated with a Dirac impulse at $t_0=0$. Initial conditions are $x_0=-0.01$ m, $v_0=0.04$ m/s. The natural frequency of the oscillator is $\omega_0 = 100$ rad/s, and the decay time is $\tau_{60} = 5$ s.}
\end{figure}


The stability of system \eqref{eq:SHOForced} may be adapted to include the effects of external forcing. We are not going to bother as much here, and we will assume that the solution will not \virg{blow up} in a finite time if the source remains bounded. 
In particular, if the source is itself bounded, we remark that integral \eqref{eq:steadyStSHO} remains bounded. 





\subsection{Response to harmonic input forcing}

Consider now a harmonic input of the form
\begin{equation}\label{eq:harmoForceLinear}
    f(t) = F e^{j\omega t},
\end{equation}
with $F \in \mathbb{C}$ being a complex forcing amplitude, and $\omega \in \mathbb{R}^+_0$ being the input forcing radian frequency (not to be confused with the natural frequency of the oscillator). We are only going to assume that the oscillator will eventually fall into a steady-state here, since from the previous discussion we know that the transient response will die out after sufficient time has elapsed, and since the forcing is of harmonic type. According to \eqref{eq:steadyStSHO}, one may compute the steady-state via the convolution integral. However, in this case one may equivalently assume that the steady state vibrates at the frequency of the input, thus
\begin{equation}
    x(t) = X e^{j\omega t},
\end{equation}
where we removed the index $st$ since we are now assuming that the transient has completely died out, and thus we can identify the whole solution $x(t)$ of \eqref{eq:SHOForced} with the steady-state. $X$ is here a complex amplitude. Substituting into the equation of motion results in
\begin{equation}\label{eq:TransXF}
    \frac{X}{F} = \frac{1}{\omega_0^2-\omega^2+2jc\omega} =  \frac{\omega_0^2-\omega^2-2jc\omega}{(\omega_0^2-\omega^2)^2+4c^2\omega^2}.
\end{equation}
Since $X,F$ are complex constants, the tangent of the phase angle  is obtained as the ratio between the imaginary and real parts, i.e. 
\begin{equation}
    \tan \left(\angle \frac{X}{F}\right) = -\frac{2c\omega}{\omega_0^2 - \omega^2}
\end{equation}
One should pay attention to the sign of the denominator when inverting the tangent function. Hence, the phase starts out at $0$ when $\omega \approx 0$; then it reaches $-\pi/2$ when $\omega \approx \omega_0$, and then approaches $-\pi$ as $\omega \rightarrow \infty$. The absolute value of the transfer function is obtained as
\begin{equation}\label{eq:XFlinearOsc}
    \left|\frac{X}{F}\right|  = \left((\omega_0^2-\omega^2)^2+4c^2\omega^2\right)^{-1/2},
\end{equation}
which has a maximum at $\omega = \omega_0 \left(1-2(c/\omega_0)^2 \right)^{1/2}$. The maximum is 
\begin{equation}
    \left|\frac{X}{F}\right|(\omega = \omega_0 \left(1-2(c/\omega_0)^2 \right)^{1/2}) \approx \frac{1}{2c\omega_0},
\end{equation}
where factors of the order $O(c^4)$ were disregarded. The \emph{bandwidth} of the transfer function is defined as the interval in frequency occurring between the frequencies $\omega_+,\omega_-$ that are found at $\frac{1}{\sqrt{2}}$ the maximum (these are the \emph{half-power points}, since power is the square of the absolute value). To obtain $\omega_\pm$, one simply uses this defintion, hence:
\begin{equation}
    \left((\omega_0^2-\omega^2)^2+4c^2\omega^2\right)^{-1/2} = \left(2\sqrt{2} c\omega_0\right)^{-1}.
\end{equation}
Solving for $\omega$, and disregarding small terms, one gets
\begin{equation}
    \omega_\pm \approx \omega_0\left(1 \pm {c} \right), \quad \rightarrow (\omega_+ - \omega_-) \approx 2c.
\end{equation}
This shows that, for small damping values, one may recover the value of the decay time from the frequency response, after measuring the bandwith of the peak in the transfer function. As a cautionary note, the transfer function $X/F$ was obtained here in the case of sinusoidal input forcing, by \virg{scanning} the frequency axis. The same transfer function may however be obtained via the impulse response \eqref{eq:GreenSHO}. To show this, it is sufficient to compute a Fourier transform and, again, we may resort to tables for this. Considering the transforms \eqref{eq:FouTtable} (with $a = \sqrt{\omega_0^2 - c^2}$, $x(t) = e^{-ct}$), one gets for the impulse response \eqref{eq:GreenSHO}
% \begin{equation}
% \mathcal{F}\left\{f(t)\sin(at)\right\}(\omega) = \frac{\hat f(\omega-a)-\hat f(\omega+a)}{2i}, \quad \mathcal{F}\left\{e^{-ct}| t \geq 0, c \geq 0\right\}(\omega) = \frac{1}{\sqrt{2\pi}(c+j\omega)}.
% \end{equation}
% Using these two identities it is possible to transform the impulse response \eqref{eq:GreenSHO}, using $f(t) = e^{-ct}$, $a = \sqrt{\omega_0^2-c^2}$. The result is
\begin{equation}
    \mathcal{F}\left\{G(t|t_0=0)\right\}(\omega) = \frac{1}{\sqrt{2\pi}}\frac{1}{(\omega_0^2-\omega^2)+2jc\omega} = \frac{1}{\sqrt{2\pi}}\frac{\omega_0^2-\omega^2-2jc\omega}{(\omega_0^2-\omega^2)^2+4c^2\omega^2},
\end{equation}
that is the same as \eqref{eq:TransXF} (up to a constant factor). Thus, knowledge of the impulse response is equivalent to \virg{scanning} the frequency axis one frequency at a time. This equivalence is often employed experimentally, where the impulse response is obtained via deconvolution of appropriate sine sweeps.


\subsubsection{Mechanical impedance}

Though \eqref{eq:TransXF} expresses the general relationship between output displacement and input forcing, it may be preferable to obtain the transfer function between output velocity and input forcing. Thus,
\begin{equation}
    \frac{dx(t)}{dt} = j\omega e^{j\omega t} \triangleq V e^{j\omega t}. 
\end{equation}
Using this in \eqref{eq:TransXF} results in
\begin{equation}
    \frac{V}{F} = \frac{j\omega(\omega_0^2 - \omega^2)-2c\omega^2}{(\omega_0^2-\omega^2)^2+4c^2\omega^2}.
\end{equation}
The ratio $V/F$ (often denoted $Y$) is the  \emph{mechanical admittance.} The inverse $F/V$ (often denoted $Z$) is the \emph{mechanical impedance.} The reason why it may be preferable to work with the impedance (or the admittance), rather than $X/F$, is that velocity and force and power-conjugated quantities: in an energy framework, knowledge of the impedance/admittance allows to describe mechanical systems in an energy-consistent manner, as we will see in due course. For the phase angle, one has
\begin{equation}
    \tan \left(\angle \frac{V}{F}\right) = \frac{\omega_0^2 - \omega^2}{(-2c\omega)}.
\end{equation}
Thus, in this case the phase angle starts out at $\pi/2$ when $\omega \approx 0$, then it goes to zero when $\omega \approx \omega_0$, and then it approaches $-\pi/2$ as $\omega \rightarrow \infty$. The absolute value is given by
\begin{equation}
    \left| \frac{V}{F} \right| = \omega \left((\omega_0^2-\omega^2)^2+4c^2\omega^2\right)^{-1/2}.
\end{equation}
Differentiating with respect to $\omega$, one gets a maximum at $\omega = \omega_0$. The maximum is 
\begin{equation}
    \left|\frac{V}{F}\right|(\omega = \omega_0) = (2c)^{-1}.
\end{equation}
\begin{figure}[hbt]
    \centering
    \includegraphics[width=0.85\linewidth]{Figures/ImpedanceSHO.png}
    \caption{Absolute values and phase angles of the transfer functions for the linear oscillator. The natural frequency of the oscillator is $\omega_0 = 100$ rad/s, the decay times are set as $\tau_{60}=\left\{1.0,1.2,1.4,1.6,1.8,2.0\right\}$ s}\label{fig:LinearTransFunctPlots}
\end{figure}



\subsubsection{Finite difference schemes}
\begin{figure}[hbt]
    \centering
    \includegraphics[width = 0.85\linewidth, clip, trim={1cm 0cm 1cm 0cm}]{Figures/SlopesForced.pdf}
    \caption{Error slopes of scheme \eqref{eq:shoFDForced}. The oscillator is activated with a Dirac delta at $t_0=0$. Initial conditions are $x_0=-0.01$ m, $v_0=0.04$ m/s. The natural frequencies of the oscillator are selected as $\omega_0 = 100$ rad/s (solid), $\omega_0 = 200$ rad/s (dashed), $\omega_0 = 300$ rad/s (dash-dotted), and the decay time is $\tau_{60} = 5$ s.}\label{fig:ErrSlopesForced}
\end{figure}
Finite difference schemes may be obtained in this case by any discretisation of $f(t)$, of the desired accuracy. For second-order accuracy, $f^n = \left\{f(t_n),\mtd f(t_n), \mtt f(t_n) \right\}$, are all valid discretisations. Hence, a suitable second-order accurate scheme is obtained as
\begin{equation}\label{eq:shoFDForced}
    \dtt x^n = -\omega_0^2 x^n -2c\dtd x^n + f^n
\end{equation}
Initialisation should be performed in such a way that second-order accuracy is preserved. In this respect, one may set $x^0 = x_0$. Then,
\eqref{eq:dtdInit} is used again, to get
\begin{equation}
    (\dtp - \frac{k}{2}\dtt)x^0 = v_0.
\end{equation}
One has
\begin{equation}
    \dtt x^0 \approx -\omega_0^2 x_0 - 2c \dtp x^0 + f^0.
\end{equation}
Using these, one can extract $x^1$ as 
\begin{equation}
    x^1 = x^0 + \frac{k v_0 + \frac{k^2}{2}\left(-\omega_0^2 x^0 + f^0 \right)}{1+kc}
\end{equation}
When approximating a Dirac delta at $t_0=0$, one should use $f^0 = 2/k$. Error slopes for scheme \eqref{eq:shoFDForced} are shown in Fig.  \ref{fig:ErrSlopesForced}.

